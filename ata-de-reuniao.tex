\documentclass[a4paper,10pt]{report}
\usepackage[brazilian]{babel}
\usepackage[utf8]{inputenc}
\usepackage[ruled,linesnumbered]{algorithm2e}
\usepackage[table,xcdraw]{xcolor}
\usepackage{graphics} 
\usepackage{epsfig} 
\usepackage{amsmath} 
\usepackage{amssymb}  
\usepackage{nicefrac}
\usepackage{subfig}
\usepackage{setspace}
\usepackage{microtype}
\usepackage{color}
\usepackage{amsthm}
\usepackage{listings,xcolor}
\usepackage{ragged2e}
\usepackage{times}
\usepackage{tabularx}
\usepackage{geometry}
\geometry{
 a4paper,
 total={170mm,257mm},
 left=20mm,
 top=20mm,
 }
\newcommand{\HRule}{\rule{\linewidth}{0.5mm}}

\begin{document} 
    \begin{center}
        \begin{large}
            \begin{figure}[ht]
            \centering
            \includegraphics[scale = 0.4]{images.tex/logo-exemplo.png}
            \end{figure}
            
            \textbf{NOME DA INSTITUIÇÃO}\\
            PROJETO: TEMA ABORDADO\\
            CURSO\\[1.5cm]
        \end{large}
    \end{center}
            
    \doublespacing
        \begin{justify}
            \begin{large}
                \textbf{ATA DA PRIMEIRA REUNIÃO ORDINÁRIA DO ANO DE DOIS MIL E VINTE UM, REFERENTE À ORGANIZAÇÃO DE PROJETO REALIZADA PELO GRUPO UM DO (CURSO) DA (INSTITUIÇÃO).}\\
            \end{large}
        \end{justify}
        
    \begin{justify}
        \begin{large}
            Aos vinte e cinco do mês de Outubro de dois mil, às vinte e três horas e trinta minutos, em salas de videoconferência, reuniu-se o Grupo 01 (um) da (Instituição), para deliberar sobre a pauta da primeira reunião ordinária de dois mil e vinte e um. Estiveram presentes os integrantes: \textbf{Primeiro Integrante}, \textbf{Segundo Integrante}, \textbf{Terceiro Integrante}, \textbf{Quarto Integrante}, \textbf{Quinto Integrante}, \textbf{Sexto Integrante} e \textbf{Sétimo Integrante}. \textbf{PAUTA: Primeiro ponto:} Apresentação geral. \textbf{Segundo ponto:} Levantamento de temas e setores de interesse. \textbf{Terceiro ponto:} Automação de Dados. \textbf{Quarto ponto:} \textit{Data Driven}. \textbf{Quinto ponto:} Integração de Relatórios. \textbf{Quinto ponto:} Encaminhamentos. No decorrer da reunião, dentre as principais temáticas abordadas na prévia, como \textit{(TEMA)} e uso de \textit{(FERRAMENTA)}, o consenso de natureza norteadora do projeto acabou voltada para o \textbf{(TEMA)}, inspirando-se em soluções como \textbf{(PRODUTO)} e \textbf{(PRODUTO)}. Em um primeiro momento, as ideias centrais elucidadas podem ser resumidas em: um sistema de \textbf{(TEMA)} com \textit{(FERRAMENTA)} e \textbf{(FERRAMENTA)}. Foram debatidos temas de menor complexidade de implementação, como o (TEMA), bem como proposições mais arrojadas que focam em \textit{(MODELO)}. Ao final da reunião colocou-se em pauta a viabilidade do referido modelo, considerando questões como (PROBLEMÁTICA) e (COMPLEXIDADE).
        \end{large}
    \end{justify}
        
\end{document}